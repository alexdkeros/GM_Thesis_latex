\chapter{Related Work} \label{chap:relWork}

A great deal of prior work exists on threshold monitoring and monitoring of distributed sets of data streams, mostly focusing on applications where the monitored function is well defined and linear. In~\cite{DilmanReactiveMonitoring} the sum of a distributed set of variables is monitored for threshold crossings, with~\cite{KeralapuraCommEfficientThresholdCounts} proposing the addition of local constraints to reduce communication overhead. Continuous tracking and approximate answering of specific aggregation operations (sum, averaging and minimum) over a coordinator-based scenario is explored in~\cite{OlstonAdaptiveFiltersContQueries}. Additionally, \cite{GibbonsSimpleFuncEstimationOverUnion, GibbonsDistrStreamAlgosSlidingWindows} provide methods for estimating simple functions over distributed data streams. $k$-largest aggregate value monitoring is described in~\cite{BabcockDistributedTopk}, where local constraint enforcement and efficient resolution of false constraint violations is presented.

Elevation of the restriction of monitoring function linearity happens at~\cite{Sharfman2006GM}, where a geometric method for threshold monitoring of arbitrary functions over coordinator-based and mesh-like network topologies is described. In~\cite{CormodeApproxContQueryingDistrStreams} approximate answers to complex aggregate queries are provided by a coordinator node, with the distributed nodes retaining synopses of the monitored data and communicating them when local constraints are being violated i.e., significant divergence of local data from the previously communicated data has been observed. Furthermore, prediction mechanisms are employed in order to reduce the communication burden. By reducing the approximate query answering to local threshold crossing monitoring at the distributed nodes,~\cite{Garofalakis2013SketchBasedGM} succeeds in unifying the continuous monitoring task with the geometric monitoring method of~\cite{Sharfman2006GM}.

Safe Zones are introduced in~\cite{Keren2013SafeZones} as an extension of the geometric monitoring method of~\cite{Sharfman2006GM},
where an arbitrary function is geometrically monitored by employing optimal local constraints at the nodes, that are fitted to each node's data distribution. In order to reduce the computational burden of optimal local constraint formation a hierarchical node clustering scheme is implemented that allows recursive computation of the problem at hand. In~\cite{Samoladas2013Unification} the Safe Zones and the bounding balls of the geometric monitoring method are proven to be fundamentally the same. Following that claim,~\cite{Sharfman2012ShapeSensGM} explores ellipsoidal bounds as a way to minimize the volume of bounding regions and reduce the communication overhead induced by false alarms. Additionally, communication of temporal data, along with first and second moments of the nodes' data distributions, as well as a method for decoupling the estimate vector from it's use as the reference vector for bounding region construction is proposed. Constraints tailored to fit data distributions at the nodes are also explored in~\cite{Keren2014GMHetStreams}, where simple and efficiently computable shapes, as well as a hierarchical clustering of the monitoring nodes into disjoint sets that maximize the probability of resolution of false alarms, are proposed. Finally,~\cite{GiatrakosPredictionGM} offers a generalization of the geometric monitoring scheme by incorporating a variety of prediction models based on velocity and acceleration of the vector representations of the data streams.

