 \chapter{Theoretical Background} \label{chap:theorBack}

The present chapter contains the background knowledge required throughout the length of this thesis. Section~\ref{sec:theorBack-GM} describes the framework of the \emph{Geometric Monitoring method}. Section~\ref{sec:theorBack-MOP} presents \emph{multi-objective optimization} and  dives into the algorithms used in our implementation. Section \ref{sec:theorBack-MWMGraphs} discusses \emph{graph maximum weight matching}, and, finally, in Section~\ref{sec:theorBack-SavitzkyGolay} we explain the \emph{Savitzky-Golay filtering} used for smoothing, velocity and acceleration approximation.

%TODO gaussian process and lwpr?

\section{Geometric Monitoring of Distributed Streams} \label{sec:theorBack-GM}
 
The \emph{Geometric Monitoring} method~\cite{Sharfman2006GM} was initially devised as a way to monitor threshold crossings of arbitrary functions over distributed data streams, i.e. be able to determine whether an arbitrary \emph{monitoring function} $f(\cdot)$ over the data streams violated a predetermined threshold ($f(\cdot)>T$ or $f(\cdot)<T$). By mapping data streams to a feature space defined by the dimensionality of each data stream update and monitoring the convex hull surrounding the value of the monitoring function, Sharfman et al. were able to decompose the monitoring task into local constraints and apply distributed threshold monitoring, while reducing or eliminating the costs required by central data processing.

In the current section a detailed presentation of this method is taking place. In Subsection~\ref{subsec:theorBack-GM-sysArch} two system architectures are shown, a fully distributed scenario and a coordinator based one, where Geometric Monitoring can be applied. Subsection~\ref{subsec:theorBack-GM-compMod} explains the computational model, followed by the method's geometric interpretation in Subsection~\ref{subsec:theorBack-GM-geomInt}. Finally, in Subsection~\ref{subsec:theorBack-GM-protocol} the application's monitoring protocol implementing the Geometric Method is described.


\subsection{System Architecture} \label{subsec:theorBack-GM-sysArch}

fully distributed node topology\\
	.no coordinator-center node\\
	.communication between nodes\\
	*image

coordinator based node topology\\
	.coordinator-center node\\
	.nodes communicate only with coordinator\\
	*image

\subsection{Computational Model} \label{subsec:theorBack-GM-compMod}

stream and node notation\\
weights\\
statistics vectors\\
	global statistics vector\\
monitored function\\
threshold

estimate vector\\
drift vector

general operation of distributed algorithm\\
	drift vector definition

general operation of coordinator based algorithm\\
	*balancing process\\
		slack vector\\
	drift vector definition

\subsection{Geometric Interpretation} \label{subsec:theorBack-GM-geomInt}

node local constraints make sure global violation is accurately monitored\\
how?

convexity property of drift vectors

theorem of bounded convex hull by local constraints (balls)

monochromaticity of balls

balls monochromatic means threshold upheld


\subsection{Protocol} \label{subsec:theorBack-GM-protocol}

decentralized algorithm (in short, for completeness)

centralized algorithm (in detail)\\
*we will focus on that


\section{Multiobjective Optimization} \label{sec:theorBack-MOP}

what is mop\\
use examples\\
kinds:\\
	a.numerical\\
	b.evolutionary
\subsection{SLSQP} \label{subsec:theorBack-SLSQP}

\subsection{Sohr's algorithm a.k.a. ralg} \label{subsec:theorBack-CONMIN}

algorithm description

\section{Savitzky-Golay Filtering} \label{sec:theorBack-SavitzkyGolay}

filtering generals\\
examples of uses of filters\\
filters:\\
	Kalman\\
		+,-
	Moving Average\\
		+,-
	Savitzky-Golay a.k.a. ???
		+,-

algorithm description

\section{Maximum Weight Matching in Graphs} \label{sec:theorBack-MWMGraphs}

general graph theory (introductory)

what is max weight matching

algorithm description
