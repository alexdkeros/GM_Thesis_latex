\documentclass[hyperref={pdfpagelabels=false}]{beamer}
\usepackage{lmodern}
\usepackage{graphicx}          	% add graphics
\usepackage{tikz,pgfplots}
\usepackage{gincltex}
\usepackage{epstopdf}
\usepackage{enumerate}		% lists of items
\usepackage[backend=biber]{biblatex}
\addbibresource{presentation.bib}




\usetheme{Dresden}
\setbeamerfont{subsection in toc}{size=\small}
\setbeamertemplate{blocks}[rounded][shadow=false]


\title{Scaling Geometric Monitoring Over Distributed Streams}  
\author{Alexandros D. Keros} 
\date{June 23, 2016} 
\addtobeamertemplate{title page}{}{\begin{center}\small Supervised by: Prof. V.Samoladas\end{center}}

\begin{document}

\begin{frame}
\titlepage
\end{frame} 


\begin{frame}
\frametitle{Table of contents}
\tableofcontents
\end{frame} 

%%%%%%%%%%%%%  section: INTRODUCTION  %%%%%%%%%%%%%%%%%%%%%%%
\section{Introduction}
\begin{frame}
  \tableofcontents[currentsection]
 \end{frame}
 
\subsection*{Overview}
\begin{frame} \frametitle{Data Stream Systems}%\footfullcite{Babcock2002DataStreamSystems}}
\begin{itemize}
\item \textbf{Data streams}: Continuous, high volume, size unbound, violative, probably distributed
\item \emph{Pull paradigm}
\item Centralizing and/or polling $\rightarrow$ prohibitive in terms of communication overhead
\item Examples: telecommunication, sensor networks 
\end{itemize}
\end{frame}

\begin{frame} \frametitle{The Geometric Monitoring Method}%\footfullcite{Sharfman2006GM}}
\begin{itemize}
\item Threshold monitoring
\item Nodes communicate when needed
	\begin{itemize}
	\item Local constraints
	\item Violation resolution (\emph{false alarms})
	\end{itemize}
\item Arbitrary function monitoring
\item Tight accuracy bounds
\item A promising framework for \emph{distributed data stream monitoring}
\end{itemize}
\end{frame}

\begin{frame} \frametitle{Motivation}
\setbeamercovered{transparent}
\begin{itemize}
\item[]<1-> Problems:
\begin{itemize}
	\item<1-> increasing node population
	\item<1-> data volume
	\item<1-> data dimensionality
	\item<1-> arbitrary functions
	\item<2-> \textbf{communication - accuracy tradeoff}
\end{itemize}
\item[]<3->Need for:
\begin{itemize}
	\item<3-> scalability warranties
	\item<3-> tight accuracy bounds
	\item<3-> incremental/real-time operation
	\item<4-> \textbf{Minimize communication while retaining accuracy bounds}
\end{itemize}
\end{itemize}
\end{frame}

\begin{frame} \frametitle{Contributions}
\setbeamercovered{transparent}
Expand the \emph{geometric monitoring method}:
\begin{itemize}
\item<1-> heuristic method for violation resolution
\item<2-> distance-based hierarchical node clustering%\footfullcite{Keren2014GMHetStreams}
\item<3-> throughout method evaluation on synthetic and real-world datasets
\end{itemize}
\end{frame}

%%%%%%%%%%%%%  section: THEOR BACK  %%%%%%%%%%%%%%%%%%%%%%%
\section{Theoretical Background}
\begin{frame}
  \tableofcontents[currentsection]
 \end{frame}
\subsection{The Geometric Monitoring Method}
\begin{frame} \frametitle{Geometric Threshold Monitoring}
\begin{itemize}
%\item \fullcite{Sharfman2006GM}
\item \textbf{Threshold monitoring}: arbitrary function $f(\cdot)$, threshold $T$\\
 $$f(\cdot)<T\ \text{or}\ f(\cdot)>T$$
\item \textbf{Idea}: decompose into local constraints at the nodes
\end{itemize}
\end{frame}

\begin{frame} \frametitle{System Architecture} \framesubtitle{Decentralized Scenario}
\begin{figure}[H]
\centering
\vspace{-0.2cm}
\includegraphics[scale=0.5]{../img/decentralized.tex}
\caption{\textbf{Mesh-like network topology} example of the decentralized scenario. Dashed lines represent data streams and half arrows represent message exchanges.} 
\end{figure}
\end{frame}

\begin{frame} \frametitle{System Architecture} \framesubtitle{Centralized Scenario}
\begin{figure}[H]
\centering
\vspace{-1cm}
\includegraphics[scale=0.5]{../img/centralized.tex}
\caption{\textbf{Star-like network topology} example of the centralized scenario.The bold node represents the coordinator node. Dashed lines represent data streams and half arrows represent message exchanges.} 
\end{figure}
\end{frame}

\begin{frame} \frametitle{Computational Model}\framesubtitle{Statistics vectors}
\setbeamercolor{block title}{use=structure,fg=white,bg=blue!55!black}
\setbeamercolor{block body}{use=structure,fg=black,bg=blue!5!white}

\begin{itemize}
\item the \emph{monitoring function} $f:\mathbb{R}^d \to \mathbb{R}$
\item the \emph{threshold} $T \in \mathbb{R}$
\item the \emph{monitoring node set} : $P=\{p_1, \dots, p_n\}$\\\quad with \emph{weights} $w_1, \dots, w_n$
\item the \emph{data streams} : $S=\{s_1, \dots, s_n\}$
\item the $d$-dimensional \emph{local statistics vectors} : $\vec{v_1}(t), \dots, \vec{v_n}(t)$\\\quad represent each node's data stream at time $t$
\end{itemize}
\begin{block}{Global statistics vector}
\vspace{0.2cm}
\begin{equation}
\vec{v}(t)=\frac{\sum_{i=1}^n{w_i\vec{v_i}(t)}}{\sum_{i=1}^n{w_i}}
\label{form:globalStatsVector}
\end{equation}
\vspace{0.2cm}
\end{block}
\end{frame}

\begin{frame} \frametitle{Computational Model}\framesubtitle{Estimate vector}
\setbeamercolor{block title}{use=structure,fg=white,bg=blue!55!black}
\setbeamercolor{block body}{use=structure,fg=black,bg=blue!5!white}
Infrequent communication between nodes/nodes-coordinator:
\begin{block}{Estimate vector}
\begin{equation}
\vec{e}(t)=\frac{\sum_{i=1}^n {w_i \vec{v_i}'}}{\sum_{i=1}^n {w_i}}
\label{form:estimateVector}
\end{equation}
\end{block}
\begin{itemize}
\item the last communicated \emph{local statistics vector} of node $p_i$ : $\vec{v_i}'$
\item \emph{Local statistics} divergence:$\Delta \vec{v_i}(t)=\vec{v_i}(t)-\vec{v_i}', i=1,\dots,n$
\end{itemize}
\begin{columns}
\setbeamercolor{block title}{use=structure,fg=blue,bg=blue!15!white}
\setbeamercolor{block body}{use=structure,fg=black,bg=blue!5!white}
\begin{column}[t]{0.5\textwidth}
\begin{block}{Decentralized drift vector}
\vspace{-0.3cm}
\begin{equation}
\vec{u_i}(t)=\vec{e}(t)+\Delta \vec{v_i}(t)
\label{form:decentralizedDrift}
\end{equation}
\end{block}
\end{column}
\begin{column}[t]{0.5\textwidth}
\begin{block}{Centralized drift vector}
\vspace{-0.45cm}
\begin{equation}
\vec{u_i}(t)=\vec{e}(t)+\Delta \vec{v_i}(t)+\frac{\vec{\delta_i}}{w_i}
\label{form:centralizedDrift}
\end{equation} 
\end{block}
\end{column}
\end{columns}
\end{frame}

\begin{frame} \frametitle{Computational Model}\framesubtitle{Balancing Process}
\setbeamercolor{block title}{use=structure,fg=white,bg=blue!55!black}
\setbeamercolor{block body}{use=structure,fg=black,bg=blue!5!white}
\begin{itemize}
\item[] \textbf{Centralized scenario}
\item[] \textbf{Purpose}: resolve possible false alarms
\item[]
\begin{block}{Balancing vector}
\begin{equation}
\vec{b}=\frac{ \sum_{p_i \in P'} {w_i\vec{u_i}(t)} }{ \sum_{p_i \in P'} {w_i} }
\label{form:balancingVector}
\end{equation}
\end{block}
\begin{itemize}
	\item the \emph{balancing set} $P'$: a subset of nodes
	\item the \emph{slack vector} at the nodes $\vec{\delta_i}=\vec{\delta_i}'+\Delta\vec{\delta_i}$, $\sum_{p_i \in P'} \Delta \vec{\delta_i}= \vec{0}$:
\end{itemize}
\begin{equation}
\Delta\vec{\delta_i}=w_i\vec{b}-w_i\vec{u_i}(t)\ \forall\ p_i \in P'
\end{equation}
, readjusts the \emph{drift vectors}~\eqref{form:centralizedDrift}.
\end{itemize}
\end{frame}

\begin{frame} \frametitle{Geometric Interpretation}\framesubtitle{Convexity Property}
\begin{block}{Convexity Property}
\begin{equation}
\vec{v}(t)=\frac{\sum_{i=1}^n {w_i\vec{u_i}(t)}}{\sum_{i=1}^n {w_i}}
\label{form:convexityProperty}
\end{equation}
\end{block}
\begin{theorem}[Sharfman et al.~\cite{Sharfman2006GM}]\label{theorem:convexHull}
Let $\vec{x}, \vec{y_1}, \dots, \vec{y_n} \in \mathbb{R}^d$ be a set of vectors in $\mathbb{R}^d$. Let $Conv(\vec{x}, \vec{y_1}, \dots, \vec{y_n})$ be the convex hull of $\vec{x}, \vec{y_1}, \dots, \vec{y_n}$. Let $B(\vec{x}, \vec{y_i})$ be a ball centered at $\frac{\vec{x}+\vec{y_i}}{2}$ and with radius of $\lVert{\frac{\vec{x}+\vec{y_i}}{2}}\rVert_2$ i.e., $B(\vec{x}, \vec{y_i})=\{\vec{z}\ |\ \lVert{\vec{z}-\frac{\vec{x}+\vec{y_i}}{2}}\rVert_2 \leq \lVert{\frac{\vec{x}+\vec{y_i}}{2}}\rVert_2 \}$, then $Conv(vec{x}, \vec{y_1}, \dots, \vec{y_n}) \subset B(\vec{x}, \vec{y_i})$.
\end{theorem}
\end{frame}

\begin{frame} \frametitle{Geometric Interpretation}\framesubtitle{Convexity Property}
\begin{figure}[H]
\begin{columns}
\begin{column}[t]{0.35\textwidth}
\vspace{-6.1cm}
\caption{Example of a convex hull (light gray) defined by the drift vectors $\vec{u_i}, i=1,2,3,4,5$. The hull is bounded by the spheres created from the estimate vector $\vec{e}$ and the drift vectors $\vec{u_i}, i=1,2,3,4,5$. The global statistics vector $\vec{v}$ is guaranteed to be contained in the convex hull of the drift vectors.} 
\end{column}
\begin{column}[t]{0.65\textwidth}
\vspace{-4cm}
\includegraphics[scale=0.35, trim=0 0 4.2cm 0]{../img/convex_hull.tex}
\end{column}
\end{columns}
\label{fig:convexHull}
\end{figure}
\end{frame}


\begin{frame} \frametitle{Geometric Interpretation}\framesubtitle{Local Constraints}

\end{frame}
\begin{frame} \frametitle{Protocol}\framesubtitle{Decentralized Algorithm}

\end{frame}
\begin{frame} \frametitle{Protocol}\framesubtitle{Centralized Algorithm}

\end{frame}
\subsection{Theoretical Tools}
\subsubsection*{Multi-objective Optimization}
\begin{frame} \frametitle{Multi-objective Optimization}

\end{frame}
\begin{frame} \frametitle{Non-linear Constraint Optimization}\framesubtitle{Primal Descent}

\end{frame}
%slsqp
\begin{frame} \frametitle{Feasible Directions}

\end{frame}
%conmin
\begin{frame} \frametitle{SQP}

\end{frame}
\subsubsection*{Savitzky-Golay Filtering}
\begin{frame} \frametitle{The Savitzky-Golay Filter}

\end{frame}
\subsubsection*{Matching in Graphs}
\begin{frame} \frametitle{Maximum Weight Matching}\framesubtitle{The Primal-Dual Method}

\end{frame}

\subsection{Related Work}
\begin{frame} \frametitle{Related Work}

\end{frame}
%%%%%%%%%%%%%  section: IMPL  %%%%%%%%%%%%%%%%%%%%%%%
\section{Problem Statement \& Implementation}
\subsection{Problem Statement}
\begin{frame} \frametitle{Problem Formulation}

\end{frame}
\subsection{Implementation}
\begin{frame} \frametitle{The Geometric Monitoring Framework}

\end{frame}
\subsubsection*{Distance-based Hierarchical Clustering}
\begin{frame} \frametitle{The Distance-based Hierarchical Clustering}\framesubtitle{The Idea}

\end{frame}
\begin{frame} \frametitle{The Distance-based Hierarchical Clustering}\framesubtitle{The Weight Function}

\end{frame}
\begin{frame} \frametitle{The Distance-based Hierarchical Clustering}\framesubtitle{The Algorithm}

\end{frame}
\subsubsection*{Heuristic Balancing}
\begin{frame} \frametitle{The Heuristic Balancing}\framesubtitle{The Idea}

\end{frame}
\begin{frame} \frametitle{The Heuristic Balancing}\framesubtitle{The Optimizing Function}

\end{frame}
\begin{frame} \frametitle{The Heuristic Balancing}\framesubtitle{The Function Formulation}

\end{frame}
\begin{frame} \frametitle{The Heuristic Balancing}\framesubtitle{The Algorithm}

\end{frame}
\subsubsection*{Violation Detection}
\begin{frame} \frametitle{Violation Detection in the Sphere}\frametitle{An Nested Optimization Problem}

\end{frame}
\subsubsection*{The Savitzky-Golay Filter}
\begin{frame} \frametitle{Velocity and Acceleration Estimation via SG Filtering}

\end{frame}
\subsubsection*{Challenges}
\begin{frame} \frametitle{Implementation Challenges}

\end{frame}
%%%%%%%%%%%%%  section: EXPERIMENTS  %%%%%%%%%%%%%%%%%%%%%%%
\section{Experimental Results}
\subsection{Data \& Setup}
\subsubsection*{Synthetic Data}
\begin{frame} \frametitle{Synthetic Data}

\end{frame}
\subsubsection*{Real-world Data}
\begin{frame} \frametitle{Real-world Data}

\end{frame}
\subsection{Experiments}
\begin{frame} \frametitle{Notation}

\end{frame}
\subsubsection*{ Matching Algorithms}
\begin{frame} \frametitle{RAND, DIST, DISTR Comparison}

\end{frame}
\subsubsection*{Balancing Methods}
\begin{frame} \frametitle{GM, HM Comparison}

\end{frame}
\subsubsection*{Synthetic Data}
\begin{frame} \frametitle{GM, HDM Comparison}\framesubtitle{Synthetic Data Monitoring}

\end{frame}
\subsubsection*{Real-world Data}
\begin{frame} \frametitle{GM, HDM Comparison}\framesubtitle{Air Pollution Monitoring}

\end{frame}
%%%%%%%%%%%%%  section: CONCLUSION  %%%%%%%%%%%%%%%%%%%%%%%
\section{Conclusions \& Future Work}
\subsection{Conclusion}
\begin{frame} \frametitle{Summary \& Concluding Remarks}

\end{frame}
\subsection{Future Work}
\begin{frame} \frametitle{Future Work}

\end{frame}

\begin{frame}[plain]
\centering
The end\\
Questions?
\end{frame}

\end{document}

